% Created 2024-07-19 Fri 18:45
% Intended LaTeX compiler: pdflatex
\documentclass[11pt]{article}
\usepackage[utf8]{inputenc}
\usepackage[T1]{fontenc}
\usepackage{graphicx}
\usepackage{longtable}
\usepackage{wrapfig}
\usepackage{rotating}
\usepackage[normalem]{ulem}
\usepackage{amsmath}
\usepackage{amssymb}
\usepackage{capt-of}
\usepackage{hyperref}

\usepackage[letterpaper, margin=1in]{geometry}

\author{Sean Ni}
\date{\today}
\title{}
\hypersetup{
 pdfauthor={Sean Ni},
 pdftitle={},
 pdfkeywords={},
 pdfsubject={},
 pdfcreator={Emacs 29.3 (Org mode 9.6.15)},
 pdflang={English}}
\begin{document}

\tableofcontents

\section{Problem 4}
\label{sec:org3965543}
Find the average of function \(f(\sigma) = |a_1 - a_2| + |a_3 - a_4| + |a_5 - a_6| + |a_7 - a_8|\), where \(\sigma\) is a permutation of \((a_1, a_2, a_3, a_4, a_5, a_6, a_7, a_8)\).



\subsection{Solution}
\label{sec:org6747ff0}
Method is to count the number of different distances noted by \(|a_i - a_j|\). If you write 1 to 8 literally as 1, 2, 3, 4, 5, 6, 7, 8, you will note that the distance is from 1 to 7. First let's count the number of distance 1.

\begin{description}
\item[{distance 1}] This is easy: pairs like \((1,2)\), \((2,3)\), \ldots{} , \((7,8)\) are distance 1. There are 7 of them.
\item[{distance 2}] \((1,3)\), \ldots{} , \((6,8)\). There are 6 of them.
\item[{distance 3}] \((1,4)\), \ldots{} , \((5,8)\). There are 5 of them.
\item[{distance 4}] \((1,5)\), \ldots{} , \((4,8)\). There are 4 of them.
\item[{distance 5}] \((1,6)\), \ldots{} , \(3,8)\). There are 3 of them.
\item[{distance 6}] \((1,7)\), \((2,8)\). There are 2 of them.
\item[{distance 7}] \((1,7)\). There are 1 of them.
\end{description}

Each pair will appear exactly this many times: \(P_6^6 \times 4 \times 2\). Explanation is: once you select a pair, e.g., \((1,2)\), you can also flip them as \((2,1)\). So that is for multiplier 2. And then you can place them in each of the 4 pairs location. And lastly, \(P_6^6\) means once you fixed the selection of the pair, the remaining 6 digits will permutate this many times. So total sum of all permutations of \(f(\sigma)\) is

\(P_6^6 * 4 * 2 * (1*7 + 2*6 + 3*5 + 4*4 + 5*3 + 6*2 + 7*1)\). Average is given by this number divided by \(P_8^8 = 40320\), which is 12.



See also: \href{https://github.com/xuanni/Problems/blob/master/problem4.cpp}{problem4.cpp} for brutal force calculation. This is to verify the above solution is correct. The output of the program is below
\begin{verbatim}
brutal force method: total = 483840, number of permutations = 40320, average = 12
analytical method: average = 12
\end{verbatim}

\section{Problem 5}
\label{sec:org938cf7c}
Find all real \(x\) such that

\begin{equation}
log_{2x}(48\sqrt[3]{3}) = log_{3x}(162\sqrt[3]{2})
\end{equation}

\subsection{Solution}
\label{sec:orga25d74d}
\begin{equation}
\frac{\ln(48\sqrt[3]{3})}{\ln(2) + \ln(x)} = \frac{\ln(162\sqrt[3]{2})}{\ln(3) + \ln(x)}
\end{equation}

\begin{equation}
\ln(x) = \frac{\ln(3)\cdot \ln(48\sqrt[3]{3}) - \ln(2)\cdot \ln(162\sqrt[3]{2})}{\ln(162\sqrt[3]{2}) - \ln(48\sqrt[3]{3})}
\end{equation}


\begin{eqnarray}
x &=& \exp\left[ \frac{\ln(3)\cdot \ln(48\sqrt[3]{3}) - \ln(2)\cdot \ln(162\sqrt[3]{2})}{\ln(162\sqrt[3]{2}) - \ln(48\sqrt[3]{3})}\right]\\
&=& \exp\left[\frac{\ln(3)\cdot\ln(3)-\ln(2)\cdot\ln(2)}{2(\ln(3)-\ln(2))}\right]\\
&=& \exp\left[\frac{\ln(6)}{2}\right]\\
x&=& \sqrt{6}
\end{eqnarray}
\end{document}
